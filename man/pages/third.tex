\section{其他工具}
\label{ch3}

\subsection{摘要}
\label{ch3_1}
\subsubsection{中文摘要}
中文摘要在''pages/chabstract.tex''文件中,内容如下
\begin{lstlisting}[frame=single,escapeinside='']
 \noindent {\bf \zihao{-4}
'论文题目:毕设模板说明书'\\
'学生姓名:王鸿鑫、秦雨果'\\
'指导教师:郝运'\\ }
\begin{chabstract}
'这是毕设模板的说明书,如果有问题,请到这里来找答案,我们希望帮助你漂亮的呈现自己的成果。'
\end{chabstract}
\chkeywords{'中文;模板;说明书'}
\end{lstlisting}
效果就如本说明书的摘要。\par

\subsubsection{英文摘要}
英文摘要在''pages/enabstract.tex''文件中,内容如下
\begin{lstlisting}[frame=single,escapeinside='']
 \noindent {\bf \zihao{-4}
Title:Template manual\\
Name:wanghongxin,qinyuguo\\
Supervisor:haoyun\\ }
\begin{enabstract}
This is the manual of the template.
\end{enabstract}
\enkeywords{English;template;manual}
\end{lstlisting}
效果如本说明书的英文摘要。\par

\subsection{附录}
\label{ch3_2}
附录文件开头部分是下面这个样子的
\begin{lstlisting}[frame=single,escapeinside='']
\appendixs{testref.bib'文件'}
\label{appen1}
\begin{lstlisting}[frame=single,escapeinside='']
@book{IEEE-1363,
  author={IEEE Std 1363-2000 and BBB and CCC and DDD},
  title={IEEE Standard Specifications for Public-Key Cryptography},
  type = {M},
  edition = "2",
\end{lstlisting}
效果如附录\ref{appen1},强调一下,附录由
\begin{lstlisting}[frame=single,escapeinside='']
\appendixs{title}
\end{lstlisting}
开始。\par

\subsection{主要符号表}
\label{ch3_3}
主要符号表由如下环境生成
\begin{lstlisting}[frame=single,escapeinside='']
\begin{denotation}
\item[label1]
\item[label2]
\end{denotation}
\end{lstlisting}
比如生成本说明主要符号表的文件``pages/denotation.tex''内容如下
\begin{lstlisting}[frame=single,escapeinside='']
\begin{denotation}
\item[HPC] '高性能计算~(High Performance Computing)'
\item[cluster] 集群
\item[Itanium] 安腾
\item[SMP] 对称多处理
\item[API] 应用程序编程接口
\item[PI]	聚酰亚胺
\item[MPI]	'聚酰亚胺模型化合物,N-苯基邻苯酰亚胺'
\item[PBI]	聚苯并咪唑
\item[MPBI]	'聚苯并咪唑模型化合物,N-苯基苯并咪唑'
\item[PY]	聚吡咙
\item[PMDA-BDA]	均苯四酸二酐与联苯四胺合成的聚吡咙薄膜
\item[$\Delta G$]  	'活化自由能~(Activation Free Energy)'
\item [$\chi$] '传输系数~(Transmission Coefficient)'
\item[$E$] 能量
\item[$m$] 质量
\item[$c$] 光速
\item[$P$] 概率
\item[$T$] 时间
\item[$v$] 速度
\item['劝  学'] '君子曰:学不可以已。青,取之于蓝,而青于蓝;冰,水为之,而寒于水。'
\end{denotation}
\end{lstlisting}

\subsection{致谢}
\label{ch3_4}
致谢由下述环境生成
\begin{lstlisting}[frame=single,escapeinside='']
\begin{mythanks}
\end{mythanks}
\end{lstlisting}
比如生成本文致谢的''pages/thanks.tex''文件内容如下
\begin{lstlisting}[frame=single,escapeinside='']
\begin{mythanks}
衷心感谢西安交通大学数学与统计学院的各位老师。对于毕业设计(论文)的指导教师,对毕业设计(论文)提过有益的建议或给予过帮助的同学、同事与集体,都应在论文的结尾部分书面致谢,言辞应恳切、实事求是。应注明受何种基金支持(没有可不写)。
\end{mythanks}
\end{lstlisting}
