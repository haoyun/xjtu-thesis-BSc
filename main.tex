%================================================
%	载入文档类,配置文档类选项
%================================================

\documentclass[%truetimes,
               %xeCJKfntef,
               print,
               %timesmath,
               amsthm,
              ]{xjtubsc}

% truetimes 选项会使用windows 自带的 Times New Roman 字体,否则使用 TeX 发行版的 Times 字体,几乎没有区别,肉眼凡胎看不出差别。开启 truetiems 选项必须使用 XeLaTeX 编译。否则关闭 truetiems。推荐开启。且推荐使用 XeLaTeX 编译。

% print 选项会去除超链接的颜色。以供打印。根据需求开启。

% timesmath 将会使用 Times 风格的数学字体,以使与正文的 Times 英文字体更加搭配。根据需求开启。

% amsthm 将会调用 amsthm 宏包及 thmtool 宏包,而且配置好了常用的 定义、定理、证明等环境。推荐开启。

%================================================
%	加载宏包、配置宏包选项、自定义命令等
%================================================


\usepackage{xjtupackages} % 这是对常用宏包的一个打包,供初级用户使用。如需在此基础上载入其他宏包,请先阅读 xjtupackages.sty 文件,以免造成冲突。

\graphicspath{{figures/},{figs/}} % 仿照此格式设置图片路径,需要graphicx支持(xjtupackages中已载入)

\DeclareMathOperator{\coker}{coker}
\newcommand\abs[1]{\left\lvert#1\right\rvert}
\newcommand*\diff{\mathop{}\!\mathrm{d}}


\begin{document}
%================================================
%	下面填写基本信息
%================================================

% 以下三项是必须的。
\author{}{}	%	作者{中文}{英文}
\title{}{}	%	题目{中文}{英文}
\advisor{}{}	%	导师{中文}{英文}

% 如果不希望 \LaTeX{} 生成任务书、评议书、意见书、答辩结果,下面的信息可以不填。这里内容换行请用\xjtunewline
\college{}	%	学院
\department{}	%	专业(系)
\class{}	%	班级
\place{}	%	毕设地点
\bdate{2012/6/6}	% 开始时间,格式为:年/月/日
\edate{2013/6/6}	% 结束时间,格式为:年/月/日

\begin{INFObackground} % 课题的背景、意义及培养目标
\end{INFObackground}

\begin{INFOdata} % 设计(论文)的原始数据与资料
\end{INFOdata}

\begin{INFOtask} % 课题的主要任务
\end{INFOtask}

\begin{INFOrequirement} % 课题的主要任务
\end{INFOrequirement}

\begin{INFOsubmit} % 课题的主要任务
\end{INFOsubmit}

\begin{INFOreference} % 课题的主要任务
\end{INFOreference}

%================================================
%	正文前
%================================================
\frontmatter

\extrapages % 用于输出任务书、评议书、意见书、答辩结果。如果不希望 \LaTeX{} 生成任务书、评议书、意见书、答辩结果,可以注释掉此句。

\begin{abstractcn} % 中文摘要



\end{abstractcn}

\keywordscn{} % 中文关键词

\begin{abstracten} % 英文摘要



\end{abstracten}

\keywordsen{} % 英文关键词

\tableofcontents % 生成目录


%================================================
%	正文
%================================================
\mainmatter


\section{}
\input{pages/section1.tex}%这里引入本章外部文件(假设在pages目录下section1.tex文件)

%================================================
%	正文后
%================================================
\backmatter

\nocite{*}
\bibliographystyle{GBT7714-2005NLang-UTF8-mod} % plain
\bibliography{ref/refs.bib}%这里引入你的参考文献文件(假设是ref目录下refs.bib文件)


\appendixs{}%附录

\begin{denotation}
\item[HPC] 高性能计算~(High Performance Computing)
\item[cluster] 集群
\item[Itanium] 安腾
\item[SMP] 对称多处理
\item[API] 应用程序编程接口
\item[PI]	聚酰亚胺
\item[MPI]	聚酰亚胺模型化合物,N-苯基邻苯酰亚胺
\item[PBI]	聚苯并咪唑
\item[MPBI]	聚苯并咪唑模型化合物,N-苯基苯并咪唑
\item[PY]	聚吡咙
\item[PMDA-BDA]	均苯四酸二酐与联苯四胺合成的聚吡咙薄膜
\item[$\Delta G$]  	活化自由能~(Activation Free Energy)
\item [$\chi$] 传输系数~(Transmission Coefficient)
\item[$E$] 能量
\item[$m$] 质量
\item[$c$] 光速
\item[$P$] 概率
\item[$T$] 时间
\item[$v$] 速度
\item[劝  学] 君子曰:学不可以已。青,取之于蓝,而青于蓝;冰,水为之,而寒于水。
\end{denotation}
%这里引入主要符号表文件(假设是pages目录下denotation.tex文件)

\begin{acknowledgment}%致谢



\end{acknowledgment}


\end{document}
