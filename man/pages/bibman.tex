\section{参考文献用法}
\label{ch4}
参考文献用下面的语法写bib文件就可以,就跟填空一样:)\par
建议在ref目录下建立bib文件,比如说我们在ref目录下建立了testref.bib文件,那么就把\mainfile中的
\begin{lstlisting}[language=tex,frame=single]
\mybibliography{}
\end{lstlisting}
大括号里填上ref/testref.bib,变成
\begin{lstlisting}[language=tex,frame=single]
\mybibliography{ref/testref.bib}
\end{lstlisting}
bib文件可以包含下面的内容
\begin{lstlisting}[escapeinside='',frame=single]
%'下面提到的项,带*的是必有的,其他的选填'

%专著
@book{citeentry, %'引用标识*'
  author={AAA and BBB}, %'主要责任者,多个人用and连接*'
  title={ASDFG}, %'文献题名*'
  type={M}, %'文献标识类型*'
  otherauthor={DDD and EEE}, %'其他负责人,多个人用and链接'
  edition={'第二版'}, %'版本项(第一版不写此项)'
  pubaddress={ZXCVB}, %'出版地'
  publisher={QWERT} %'出版者*'
  pubyear=2000, %'出版年*'
  pages={100-200}, %'引文页码'
  theurl={QWER} %'获取和访问路径'
  lang = {zh} %'语言项,中文写{zh},外文不写此项'
}

%专著中的析出文献
@inbook{citeentry, %'引用标识*'
  inauthor={AAA and BBB}, %'析出文献主要责任者,多个人用and连接*'
  intitle={ASDFG}, %'析出文献题名*'
  type={M}, %'文献标识类型*'
  inotherauthor={DDD and EEE}, %'析出文献其他负责人,多个人用and链接'
  author={FFF},  %'专著主要责任者,多个人用and连接*'
  title={NMKL}, %'专著题名*'
  othertitle={UIOP}, %'其他题名信息'
  edition={'第二版'}, %'版本项(第一版不写此项)'
  pubaddress={ZXCVB}, %'出版地'
  publisher={QWERT} %'出版者*'
  pubyear=2000, %'出版年*'
  pages={100-200}, %'析出文献起止页码'
  theurl={QWER} %'获取和访问路径'
  lang = {zh} %'语言项,中文写{zh},外文不写此项'
}

%连续出版物
@countinue{citeentry, %'引用标识*'
  author={AAA and BBB}, %'主要责任者,多个人用and连接*'
  title={ASDFG}, %'文献题名*'
  othertitle={YUIO}, %'其他题名信息'
  type={M}, %'文献标识类型*'
  startyear={1234}, %'起始年*'
  startvolume={90}, %'起始卷'
  endyear={1245}, %'终止年'
  endvolume={99}, %'终止卷'
  pubaddress={NJIUHY}, %'出版地'
  publisher={QWERT} %'出版者*'
  theurl={QWER} %'获取和访问路径'
  lang = {zh} %'语言项,中文写{zh},外文不写此项'
}

%连续出版物中的析出文献
@countinue{citeentry, %'引用标识*'
  inauthor={AAA and BBB}, %'析出文献主要责任者,多个人用and连接*'
  intitle={ASDFG}, %'析出文献题名*'
  type={M}, %'文献标识类型*'
  title={KJHGUU}, %'连续出版物题名*'
  othertitle={YUIO}, %'其他题名信息'
  startyear={1234}, %'年*'
  startvolume={90}, %'卷'
  pages={124}, %'页码'
  theurl={QWER} %'获取和访问路径'
  lang = {zh} %'语言项,中文写{zh},外文不写此项'
}

%专利文献
@patent{citeentry, %'引用标识*'
  author={AAA and BBB}, %'专利发明者,多个人用and连接*'
  otherauthor={DDD and EEE}, %'专利申请者或所有者,多个人用and链接'
  title={ASDFG}, %'专利题名*'
  nation={China}, %'专利国别*'
  number={123456}, %'专利号*'
  type={M}, %'文献标识类型*'
  date={2013}, %公告日期或公开日期
  theurl={QWER} %'获取和访问路径'
  lang = {zh} %'语言项,中文写{zh},外文不写此项'
}

%电子文献
@emedia{citeentry, %'引用标识*'
  author={AAA and BBB}, %'主要责任者,多个人用and连接*'
  title={ASDFG}, %'题名*'
  othertitle={TGBJ}, %'其他题名信息'
  type={M}, %'文献标识类型*'
  mediatype={CD}, %'载体类型标识*'
  pubaddress={ZXCVB}, %'出版地'
  publisher={QWERT} %'出版者*'
  pubyear=2000, %'出版年*'
  date={1234}, %'更新或修改日期'
  theurl={QWER} %'获取和访问路径'
  lang = {zh} %'语言项,中文写{zh},外文不写此项'
}

\end{lstlisting}
编译的时候先对\mainfile执行xelatex,再对\mainfile文件执行bibtex,再对\mainfile执行两遍xelatex。\par
附录\ref{appen1}中我们列出了生成本文参考文献的testref.bib,需要的同学可以参考。\par
文中引用的时候使用
\begin{lstlisting}[frame=single]
\cite{}
\end{lstlisting}
就可以了。\par
下面是一段引用的例子:\par
本模板推荐使用~BIB\TeX,样式文件为~thubib.bst,基本符合学校的参考文献格式(如专利
等引用未加详细测试)。看看这个例子,关于书的\cite{tex, companion, IEEE-1363},还有这些 \cite{Krasnogor2004e, clzs, zjsw},关于杂志的\cite{ELIDRISSI94,  MELLINGER96},硕士论文...\par